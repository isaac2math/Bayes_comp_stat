\usepackage{amsmath}
\usepackage{amssymb}
\usepackage{amsfonts}
\usepackage{euscript}
\usepackage{amsthm}
\usepackage{mathrsfs}
\usepackage{enumitem}
\usepackage{algorithm}
\usepackage[noend]{algorithmic}
\usepackage[usenames]{xcolor}
\usepackage{mdframed}
\usepackage{tcolorbox}
\usepackage{multicol}
\usepackage{dsfont}
\usepackage{adjustbox}
%\usepackage{undertilde}

%%%%%%%%%%%  Environment styles  %%%%%%%%%%%%%%%%
\newtheoremstyle{theoremStyle}% name of the style to be used
  {\topsep}% measure of space to leave above the theorem. E.g.: 3pt
  {\topsep}% measure of space to leave below the theorem. E.g.: 3pt
  %%{\itshape}% name of font to use in the body of the theorem
  {}
	{0pt}% measure of space to indent
  {\bfseries}% name of head font
  {}% punctuation between head and body
  { }% space after theorem head; " " = normal interword space
  {\thmname{#1}\thmnumber{ #2}\thmnote{ (#3)}.}
	%%theorem
\theoremstyle{theoremStyle}
\newtheorem{mythm}{Theorem}[section]

\theoremstyle{theoremStyle}
\newtheorem{mylem}{Lemma}[section]
\theoremstyle{theoremStyle}
\newtheorem{mycor}{Corollary}[section]
\newtheorem{myprop}{Proposition}[section]

\newtheoremstyle{defStyle}% name of the style to be used
  {\topsep}% measure of space to leave above the theorem. E.g.: 3pt
  {\topsep}% measure of space to leave below the theorem. E.g.: 3pt
  {}% name of font to use in the body of the theorem
  {0pt}% measure of space to indent
  {\bfseries}% name of head font
  {.}% punctuation between head and body
  { }% space after theorem head; " " = normal interword space
  {\thmname{#1}\thmnumber{ #2}\thmnote{ (#3)}}
	%%theorem
\theoremstyle{defStyle}
\newtheorem{mydef}{Definition}[section]


\newtheoremstyle{proofStyle}% name of the style to be used
  {\topsep}% measure of space to leave above the theorem. E.g.: 3pt
  {\topsep}% measure of space to leave below the theorem. E.g.: 3pt
  {}% name of font to use in the body of the theorem
  {0pt}% measure of space to indent
  {\itshape}% name of head font
  {.}% punctuation between head and body
  { }% space after theorem head; " " = normal interword space
  {\thmname{#1}}
	%%theorem
\newtheoremstyle{ExampleStyle}% name of the style to be used
  {\topsep}% measure of space to leave above the theorem. E.g.: 3pt
  {\topsep}% measure of space to leave below the theorem. E.g.: 3pt
  {}% name of font to use in the body of the theorem
  {0pt}% measure of space to indent
  {\bfseries}% name of head font
  {.}% punctuation between head and body
  { }% space after theorem head; " " = normal interword space
  {\thmname{#1}}
	%%theorem

%%%%%%%%%%%%%%%  Box texts %%%%%%%%%%%%%%%%%%%%
%\definecolor{mycolor}{rgb}{0.122, 0.435, 0.698}% Rule colour
%{
%\makeatletter
%\newcommand{\mybox}[1]{%
  %\setbox0=\hbox{#1}%
  %\setlength{\@tempdima}{\dimexpr\wd0+13pt}%
  %\begin{tcolorbox}[colframe=mycolor,boxrule=0.5pt,arc=4pt,
      %left=6pt,right=6pt,top=6pt,bottom=6pt,boxsep=0pt,width=\@tempdima]
    %#1
  %\end{tcolorbox}
%
%\makeatother
%}

%%%%%%%%%%%%%%%%  Environmnts %%%%%%%%%%%%%%%
\theoremstyle{proofStyle}
\newtheorem{myproof}{Proof}
\theoremstyle{ExampleStyle}
\newtheorem{myexample}{Examples}
\theoremstyle{theoremStyle}
\newtheorem{myexercise}{Excercise}[section]
\theoremstyle{theoremStyle}
\newtheorem{myquestion}{Question}
\theoremstyle{proofStyle}
\newtheorem{mysol}{Solution}
\newtheorem{myremark}{Remark}


%%%%%%%%%%%%%%%%%%%%  Macros  %%%%%%%%%%%%%%%%%%%%%%%%%

\newcommand{\bp}{\begin{myprop}}
\newcommand{\ep}{\end{myprop}}
\newcommand{\bpr}{\begin{myproof}}
\newcommand{\epr}{\end{myproof}}
\newcommand{\bl}{\begin{mylem}}
\newcommand{\el}{\end{mylem}}
\newcommand{\bc}{\begin{mycor}}
\newcommand{\ec}{\end{mycor}}
\newcommand{\bt}{\begin{tcolorbox}[colback=green!5, colframe=green!30!brown]\begin{mythm}}
\newcommand{\et}{\end{mythm}\end{tcolorbox}}
\newcommand{\bd}{\begin{tcolorbox}[colback=blue!2, colframe=lightgray!50!purple]\begin{mydef}}
\newcommand{\ed}{\end{mydef}\end{tcolorbox}}
\newcommand{\bex}{\begin{myexample}}
\newcommand{\eex}{\end{myexample}}
\newcommand{\bx}{\begin{tcolorbox}[colback=orange!5, colframe=pink]\begin{myexample}}
\newcommand{\ex}{\end{myexample}\end{tcolorbox}}
\newcommand{\bxc}{\begin{myexercise}\mbox{}}
\newcommand{\exc}{\end{myexercise}}
\newcommand{\bs}{\begin{mysol}}
\newcommand{\es}{\end{mysol}}
\newcommand{\br}{\begin{myremark}}
\newcommand{\er}{\end{myremark}}
\newcommand{\bq}{\begin{myquestion}}
\newcommand{\eq}{\end{myquestion}}

\newcommand{\bpic}[2]{\begin{minipage}[t]{\linewidth}\adjustbox{valign=t}{\includegraphics[width= #1\textwidth]{#2}}\end{minipage}}


\newcommand{\bn}{\begin{tcolorbox}[colback=blue!2, colframe=lightgray!50!purple]}
\newcommand{\en}{\end{tcolorbox}}

%%math operators
\DeclareMathOperator{\Ima}{Im}
%%\DeclareMathOperator{\max}{max}
%%\DeclareMathOperator{\deg}{deg}
\newcommand{\norm}[1]{\left\lVert #1\right\rVert}
\newcommand{\tr}{\mbox{tr}}
\newcommand{\inprod}[2]{\left\langle #1,#2\right\rangle}

%% fonts %%%
\newcommand{\cm}[1]{\CMcal{#1}}
\newcommand{\scr}[1]{\mathscr{#1}}
\newcommand{\bb}[1]{\mathbb{#1}}

%% sets %%
\newcommand{\R}{\bb R}
\newcommand{\N}{\bb N}
\newcommand{\Z}{\bb Z}
\newcommand{\C}{\bb C}
\newcommand{\seq}[2]{\left({#1}\right)_{#2 = 1}^\infty}
\newcommand{\seqq}[3]{\left({#1}\right)_{#2}^{#3}}
\newcommand{\sigA}{\sigma\mbox{-algebra}}

%%vectors and matrices
\renewcommand{\v}[1]{\boldsymbol{#1}}
\newcommand{\m}[1]{\mathrm{#1}}
\newcommand{\vt}[1]{\utilde{#1}}
\renewcommand{\vec}[1]{\overset{\longrightarrow}{#1}}


%%Prob and stats
\newcommand{\Em}{\mathbb{E}}
\newcommand{\Prob}{\mathbb{P}}
\newcommand{\Var}{\mathbb{V}ar}
\newcommand{\Cov}{\mathbb{C}ov}
\newcommand{\muhat}{\hat{\mu}}
\newcommand{\Ndist}{\CMcal{N}}

%%operators
\newcommand{\limitinf}[1]{\underset{#1\rightarrow\infty}{\lim\inf\,}}
\newcommand{\Limitinf}[2]{\underset{#1=1}{\overset{\infty}{\bigcup}}\underset{#2=#1}{\overset{\infty}{\bigcap}}}
\newcommand{\limitsup}[1]{\underset{#1\rightarrow\infty}{\lim\sup\,}}
\newcommand{\Limitsup}[2]{\underset{#1=1}{\overset{\infty}{\bigcap}}\underset{#2=#1}{\overset{\infty}{\bigcup}}}
\newcommand{\cupft}[3]{\underset{#1=#2}{\overset{#3}{\bigcup}}}
\newcommand{\capft}[3]{\underset{#1=#2}{\overset{#3}{\bigcap}}}
\newcommand{\capp}[2]{\underset{#1}{\overset{#2}{\bigcap}}}
\newcommand{\cupp}[2]{\underset{#1}{\overset{#2}{\bigcup}}}
\newcommand{\caps}[1]{\underset{#1}{\bigcap}}
\newcommand{\cups}[1]{\underset{#1}{\bigcup}}
\newcommand{\summ}[2]{\underset{#1}{\overset{#2}{\sum}}}
\newcommand{\sums}[1]{\underset{#1}{\sum}}
\newcommand{\limm}[2]{\underset{#1\rightarrow #2}{\lim}}
\newcommand{\prodd}[2]{\underset{#1}{\overset{#2}{\prod}}}
\newcommand{\prods}[1]{\underset{#1}{\prod}}

\DeclareMathOperator{\re}{Re}
\DeclareMathOperator{\im}{Im}
\DeclareMathOperator{\Arg}{Arg}

%%calculus
\newcommand{\dd}{\mathrm{d}}
\newcommand{\di}{\;\mathrm{d}}

%%functions
\newcommand{\ind}{\textbf{1}}

%%teaching
\newcommand{\dline}{

\vspace{2mm}

\dotfill}